%Anie - ohkis.sourceforge.net
%Unai Martinez Corral
%umartinez012@ikasle.ehu.es
%
% <- anie.tex

\section{Lizentziak eta aitorpenak}

Txantiloi hau hurrengo lizentziaren arabera eskaintzen da:

\begin{center}
\large\bfseries
Creative Commons Attribution-ShareAlike 3.0\\(CC BY-SA 3.0)
\end{center}

\begin{itemize}
\item{Egin ditzakezunak:
 \begin{description}
  \item[Banatzea]{Kopiatu, banatu eta hedatzea}
  \item[Moldatzea]{Lana egokitzea eta eratorriak egitea}
  \item[lana merkataritza helburuekin erabiltzea]{}
 \end{description}
}
\item{Hurrengoak bete bitartean:
 \begin{description}
  \item[Aitortzea]{Lanaren iturria aitortu behar da, \emph{Unai Martinez Corral} eta \emph{ITSAS}i erreferentzia eginez, eta \url{itsas.ehu.es/workgroups/latex} orrialdea aipatuz (baina lan eratorriek edo lanaren erabilerek hauen babesa dutela adierazi barik).}
  \item[Berdin partekatzea]{Lan hau moldatu edo egokituz gero, edo lan eratorririk sortzekotan, egindakoa banatzeko honetan erabilitako lizentzia berdina erabili behar da.}
 \end{description}
}
\end{itemize}

\begin{center}
\large\url{creativecommons.org/licenses/by-sa/3.0/es/legalcode.eu}
\end{center}

\bigskip
\bigskip

Aipatutako ITSASen lan taldeetako baliabideez gain, jarraian zerrendatutakoak erabili dira:

\begin{itemize}
\item{\textbf{TeXmaker} (\url{xm1math.net/texmaker}) \emph{Pascal Brachet}}
\item{\textbf{BibTeX} (\url{bibtex.org}) \emph{Oren Patashnik, Leslie Lamport, Oren Patashnik}}
\item{\LaTeX{} paketeak (\url{ctan.org/pkg/}):
 \begin{description}
 \item[import]{\emph{Donald Arseneau}}
 \item[inputenc]{\emph{Alan Jeffrey, Frank Mittelbach}}
 \item[babel]{\emph{Javier Bezos, Johannes L. Braams}}
 \item[geometry]{\emph{Hideo Umeki}}
 \item[graphicx]{\emph{David Carlisle}}
 \item[natbib]{\emph{Patrick W. Daly, Arthur Ogawa}}
 \item[caption]{\emph{Axel Sommerfeldt}}
 \item[indentfirst]{\emph{Davis Carlisle}}
 \item[multirow]{\emph{Piet van Oostrum, Jerry Leichter}}
 \item[amsmath]{\emph{The American Mathematical Society}}
 \item[eurofont]{\emph{Rowland McDonnell}}
 \item[xcolor]{\emph{Uwe Kern}}
 \item[listings]{\emph{Brooks Moses, Carsten Heinz}}
 \item[tikz,pgfplots]{\emph{Till Tantau, Christian Feuers{\"{a}}nger}}
 \item[tikz-timing]{\emph{Martin Scharrer}}
 \item[url]{\emph{Donald Arseneau}}
 \item[hyperref]{\emph{Heiko Oberdiek, Sebastian Rahtz}}
 \item[etoolbox]{\emph{Philipp Lehman}}
 \item[minitoc]{\emph{Jean-Pierre Drucbert}}
 \item[eso-pic]{\emph{Rolf Niepraschk}}
 \item[fancyhdr]{\emph{Piet van Oostrum}}
 \end{description}
}
\item{\textbf{QtikZ} (\url{hackenberger.at/blog/ktikz-editor-for-the-tikz-language}) \emph{Florian Hackenberger}}
\item{TikZ irudiak (\url{texample.net/tikz/examples/}):
 \begin{description}
  \item[Gajski-Kuhn Y-chart]{\emph{Ivan Griffin}} 
  \item[Control system principles]{\emph{Kjell Magne Fauske}}
  \item[Timing diagram with the tikz-timing package]{\emph{Martin Scharrer}}
 \end{description}
}
\end{itemize}

%\begin{center}
% \includegraphics[width=200pt]{./images/ccby.png}
%\end{center}
%
%Dokumentu hauek guztiak, iturri kodea izan ezik, Creative Commons By 3.0 (CC-by-3.0) lizentiak babesten ditu. Kopia, banaketa eta jendaurrean aurkeztea baimenduta daude, baita aldatzea eta moldatzea ere, betiere Unai Martinez Corral aipatuz autoretza aitortzen bada (baina ez honen, autorearen, babesa duela edo bere lanaren erabilera babesten duela aditzera emanez).
%
%\href{http://creativecommons.org/licenses/by/3.0/legalcode}{Lege testu} osoa \href{http://creativecommons.org}{Creative Commons} erakundearen web gunean dago eskuragarri::
%\begin{center}http://creativecommons.org/licenses/by/3.0/legalcode\end{center}
%
%\vspace{1cm}
%
%\begin{center}
% \includegraphics[width=100pt]{./images/gpl.png}
%\end{center}
%
%Programaren iturri kodea (bai mikrokontroladorerako C lengoiaian idatzitakoa, bai LabVIEW-rako era grafikoan egindakoa) GPL lizentziak babesten du, fitxategien goiburuek adierazten duten bezala. \ref{gpl} eranskinean lege testu osoa dago eskuragarri. \href{http://www.gnu.org}{GNU Proiektua}-ren web gunean ere aurkitu daiteke.